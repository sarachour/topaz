\documentclass[11pt, oneside]{article}   	% use "amsart" instead of "article" for AMSLaTeX format
\usepackage{geometry}                		% See geometry.pdf to learn the layout options. There are lots.
\geometry{letterpaper}                   		% ... or a4paper or a5paper or ... 
%\geometry{landscape}                		% Activate for for rotated page geometry
%\usepackage[parfill]{parskip}    		% Activate to begin paragraphs with an empty line rather than an indent
\usepackage{graphicx}				% Use pdf, png, jpg, or eps§ with pdflatex; use eps in DVI mode
								% TeX will automatically convert eps --> pdf in pdflatex		
\usepackage{amssymb}
\usepackage{listings}
\title{Optimization Formalization}
\author{Sara Achour}
%\date{}							% Activate to display a given date or no date


\lstset{
	basicstyle=\fontsize{8}{11}\selectfont\ttfamily,
	breaklines=true
}
\begin{document}
\maketitle
%\section{}
%\subsection{}
\section {Model V2}
\subsection {Hardware Model}
\begin{lstlisting}[mathescape]
Worker Machine $m \in M, |M|$
Main machine $R$

Machine cost for task with runtime t:

$c = a \cdot W \cdot t + b \cdot E \cdot t$
	E: energy per unit time.
	W: amount of work per unit time.
	a: weight to apply to machine efficiency.
	b: weight to apply to machine power consumption.


$p_{fail}$ : probability of a crash per unit time.
$p_{hang}$ : probability of a hang per unit time.
$p_{err}$ : probability of soft data error occurring per unit time.
$p_{ok} = 1-p_{fail} + p_{hang} + p_{err}$

Oversimplification: Assuming the probability of various failures 
is not dependent on the computation running.
\end{lstlisting}
\subsection {Programming Model}
\begin{lstlisting}[mathescape]
taskset T:
   size: |T|
   number inputs: n
   number outputs: m
   
task $t \in T$:
   $t: R^m \rightarrow R^n$
      task t is over real numbers. maps m dimension 
      vector to n dimension vector.
   $t: f(i_1, \cdot, i_n)\rightarrow (r_1, \cdot, r_m)$
      i_k : kth input, let $\vec i$ be input vector.
      r_k : kth output, let $\vec r$ be output vector.
   $P(\vec r, \vec i)$
      The probability distribution of the outputs and inputs.
      $t: f(i_1, \cdot, i_n)\sim argmax_{\hat r} P(\hat r|\hat i)$
         The input-output relation f can be rewritten as the expectation maximization of r given i.
         Since f is deterministic, there should be a particular r where P(r,i) = 1 for every i.
      ex: $P(\vec r | \vec i)$
         The probability of outputs r occurring, given input i.
   $\delta_t$: the runtime of the particular task. 
\end{lstlisting}
\subsection {Hardware-Programming Model Interaction}
\begin{lstlisting}[mathescape]
For a particular task t running on machine m:
	$p_{fail}^{t} = \delta \cdot p_{fail}$
	$p_{hang}^{t} = \delta \cdot p_{hang}$
	$p_{err}^{t} = \delta \cdot p_{err}$
	$P_{err}(\vec r, \vec i)$
		This is the output distribution of the task if an error occurs.
	$P'(\vec r, \vec i) =P(\vec r, \vec i) + P_{err}(\vec r, \vec i)$
		the output distribution on machine m (P') is a mixture of the  
		output distribution (P) distribution and the error distribution($P_{err}$).
	$P_{\epsilon}(\epsilon)$
		this is the distribution of errors between the correct, faulty result. Note: we may want to
		optimize out this distribution, but for simplicity, I will make this directly accessible.
		
Question: given result $\vec r \sim P'(\vec r, \vec i)$ and known $\vec i$ is it drawn from $P(\vec r, \vec i)$ or $P_{err}(\vec r, vec i)$?

	if:
		$P(\vec r, \vec i) > P_{err}(\vec r, \vec i) \rightarrow P$
	otherwise
		$\rightarrow P_{err}$
	
Now suppose we do not have $\vec i$. We find which distribution has the highest probability of producing $\vec r$ across all inputs.

	let $P(\vec r) =  \int_{\vec i} P(\vec r, \vec i) P(\vec i)$
	let $P_{err}(\vec r) =  \int_{\vec i} P_{err}(\vec r, \vec i) P(\vec i)$
	if
		$P(\vec r) > P_{err}(\vec r)\rightarrow P(\vec r, \vec i)$
	otherwise
		$\rightarrow P_{err}$
	


Aside:
We don't have the actual probability distributions: $P, P_{err}$, we have incomplete approximations $\hat P, \hat P_{err}$ drawn from a series of observations $(r_{corr}, r_{err}) \in O$. 

* We can pay to figure out which distribution the result comes from and improve estimates $\hat P$, $\hat P_{err}$. 
\end{lstlisting}
\subsection {Cost Function}
\begin{lstlisting}[mathescape]
# Accuracy Requirement
$P(|f(\vec i) - f'(\vec i)| < \epsilon) > \omega$
	$\epsilon$: tolerated error
	$\omega$ : tolerated frequency of intolerable errors.
	
# Relevant machine parameters:
	$P(|f(\vec i) - f'(\vec i)| < \epsilon) \approx  p_{err} \int_{0}^{\epsilon} P_{\epsilon}(\epsilon) d \epsilon$
	Compute the probability of an error occuring times the probability the error is less than epislon.

# Semantics for cost function
	To differentiate between machines $m \in M$ : $\{ m_1, \cdots, m_n\} = M$
		$m_i \rightarrow \{P_{\epsilon}^i, p_{fail}^i, p_{hang}^i, p_{err}^i, c^i\}$
	To differentiate between tasks $t \in T$: $\{t_1, \cdots, t_n\} = T$
		There are no particular parameters to annotate, we assume we're talking about one taskset.
	The reliable machine has the following properties:
		$r \rightarrow \{P_{\epsilon}^r = 0, p_{fail}^r =0, p_{hang}^r = 0, p_{err}^r=0\}$


# Accuracy Requirement for a task-machine mapping
   Given we have assignments, $t_i \in T | t_i \rightarrow m_j$:
   	$P(|f(\vec i) - f'(\vec i)| < \epsilon)  = \frac 1 {|T|} \sum_{t_i=t_0}^{t_n} P_{m_j}(|f(\vec i) - f'(\vec i)| < \epsilon)$
   		The probability of a set of task-machine assignments is the average of each probability of each assignment meeting the assignment.

# Minimization formula
	For a set of assignments $t_i \in T | t_i \rightarrow m_j$, we want to maximize:
		$\sum_{t_i \in T} c_j$
		
# Putting it together:
	We want to find a set of $t_i \rightarrow m_j$ mappings that minimizes:
		$\sum_{t_i \in T} c_j$
	but meets the following constraint:	
		$[\frac 1 {|T|} \sum_{t_i=t_0}^{t_n} P_{m_j}(|f(\vec i) - f'(\vec i)| < \epsilon)] > \omega$
		ie:
		$[\frac 1 {|T|} \sum_{t_i=t_0}^{t_n} p_{err}^j \int_{-\epsilon}^{\epsilon} P_{\epsilon}^j(\epsilon) d \epsilon] > \omega$
		
#Simple case 1: Unreliable machine, reliable machine, T tasks, no re-execution  		
   Assume, $M = {m_0}$ and $c_0 < 1, c_r = 1$. The minimization problem becomes run as many tasks on the unreliable machine.
	$[\frac 1 {|T|} \sum_{t_i=t_0}^{t_n} p_{err}^j \int_{-\epsilon}^{\epsilon} P_{\epsilon}^j(\epsilon) d \epsilon] > \omega$
	We can flip the problem to consider the probability of a bad error occuring:
	$[\frac 1 {|T|} \sum_{t_i=t_0}^{t_n} p_{err}^j (1-\int_{-\epsilon}^{\epsilon} P_{\epsilon}^j(\epsilon) d \epsilon)] < 1 - \omega$
	lets say we execute k tasks on the unreliable processor:
	$[\frac k {|T|} p_{err}^j (1-\int_{-\epsilon}^{\epsilon} P_{\epsilon}^j(\epsilon) d \epsilon)] < 1 - \omega$
  	$k  < \frac {|T| (1 - \omega)} {p_{err}^j (1-\int_{-\epsilon}^{\epsilon} P_{\epsilon}^j(\epsilon) d \epsilon)}$
  	
  	We can run a maximum of k tasks on the unreliable machine where k is
  	$k  < \frac {|T| (1 - \omega)} {p_{err}^j (1-\int_{-\epsilon}^{\epsilon} P_{\epsilon}^j(\epsilon) d \epsilon)}$
  	
\end{lstlisting}


\subsection{Outlier Detector}
\begin{lstlisting}[mathescape]
	Suppose we have an outlier detector $P_{out}(\vec r)$:
		if $P_{out}(\vec r) \neq 0 \rightarrow ok$
		else $\rightarrow outlier$
		
# Outlier Detector
	Suppose the outlier detector perfectly captures the reliable result probability distribution: 
		$P_{out}(\vec r) = P(\vec r) = \int_{i} P(\vec r, \vec i) P(\vec i) di$
	
# Impact on Error
	The maximum error is bounded by
		$\epsilon_{max} = argmax_{\vec r_a, \vec r_b} ~~ \sqrt{(\vec r_a - \vec r_b)^2} ~~ | ~~ P_{out}(\vec r_a) > 0 ~and~ P(\vec r_b) > 0 $
	
	Let $f_{e}(\epsilon) = \{1 ~ when ~ |\epsilon| \leq \epsilon_{max}, 0 ~ otherwise\}$. 
	
	The new error distribution with this outlier detector is:
		$P_e^{j\prime}(\epsilon) = P_e^{j}(\epsilon) \cdot f_{e}^j(\epsilon)$ 
	

# Impact on Cost
	The probability of re-execution is
	    let $g_{e}(\vec r) = \{0~if~P_{out}(\vec r) > 0, ~ 1 ~ otherwise\}$
		$p_{reexec}^j = \int_{-\infty}^{\infty} P{j'}(\vec r) \cdot g_{e}(\vec r)$
	
	So the added predicted cost for machine j is
		$c_r \cdot p_{reexec}^j + c_j$
\end{lstlisting}
\subsection{Martin Notes}

\begin{lstlisting}[mathescape]
================================================
Add probability outlier detector detects outlier (increase cost probabilistically, improve accuracy probabilistically)
=================================================
	Martin:
	* 2 different probability distributions, each output comes from a mixture of the two. (statistical inference, 		figure out likelihood) error prob distribution, and non-error distribution
	optimization problem: pay cost to get truth.
	   
	* joint probability distribution of the inputs and the outputs (deterministic). Project the dots down
	onto the outputs (integration), figure out which distribution the output comes from.
\end{lstlisting}




\section {Model}
\textbf {Machine Model}\\
Worker Machine $m \in M, |M|$\\
Main machine $R$\\
Each worker machine has cost: \\
$c = a \cdot work \cdot t + b \cdot energy \cdot t$\\\\
energy: energy per time.\\
work: amount of work per time $t={t_a, t_b}$.\\
For each task T, the machine has the following properties:\\
For $(m, T)$\\
$P_{crash}$: The per-task runtime probability distribution.\\
$P_{hang}$: The per-task runtime probability distribution.\\
$P_{r_i}(x|r,t)$: The probability distribution of results produced by this machine, given the correct output $r$ and the runtime.\\ 
Note: These properties are not known by topaz\\
\\
\textbf {Programming Model}\\
For taskset T: $t \in T, |T|$\\
Task inputs and outputs: $r:R^n \rightarrow R^m ~|~ t(i_1, i_2 \cdots) \rightarrow (r_1, r_2, \cdots)$\\
The inputs are real values. A fixed array of size n is interpreted as n unit values.\\
each task has a runtime: t\\

\textbf{Observed Properties}\\\\
For each machine:\\
For $(m, T) | m \in M \rightarrow (\{OD_{r_i}\}, OD_{time})$
\\
each task has a time that it takes to execute on some machine
$\hat P_{r_i}(x)$: For each result $r_i$, the observed output probability distribution. Used to determine if a runtime is reasonable.\\
$\hat P_{time}$: The observed distribution of runtimes. Used to determine if a runtime is reasonable.\\\\
\textbf {Scheduling Tasks}\\
${t_1, t_2, ...} \rightarrow m$\\
topaz may send any number of tasks at once to machine m.
\section {Programming Model}
\section {Notation}
Tasks: $t_i \in T$, $|T| = n$\\\\
Machine: $m_j \in M, |M| = m$, $r: reliable$\\\\
Cost: $m_j^w \in M ~|~ c_j = a \cdot \frac 1 {speed_j} + b \cdot {energy_j}$\\\\
\textit{Martin: each machine has a speed and energy. Each task has an amount of work. The amount of time it takes machine to execute task i is $work_i \cdot speed_j$.}\\
Topaz executes on a distributed system with m worker machines, $m_i^w \in M$, $|M| = m$ and a reliable main machine $m^r$. Each worker machine has a cost associated with it $c_i$, the reliable machine has a cost $c_r$.\\
\textbf{Optimization Problem:} we want to minimize the cost of the execution of the taskset, C, while meeting the quality of result, Q.
\subsection {Cost}
The cost of a machine $c_i = f(perf,energy)$, where $c_i \propto \frac 1 {perf}$ and $c_i \propto energy$. We can express $c_i = a \cdot \frac 1 {perf_i} + b \cdot energy_i$, where the weights may be user specified. I separate the cost from properties of the machine to keep the cost model sufficiently abstract.\\\\
\textbf{Assumption:} \textit{The reliable machine is more expensive than any of the other machines in the cluster} $\forall c_i : m_i \in M; c_r > c_i$
\subsection {Correctness}
We need some way of quantifying the quality of the result. Given a particular solution, each machine generates a result from a distribution centered around the solution. For example, the reliable machine generates the solution with $P(Sol) = 1$. For now on $s_i$ is the correct solution for task i.
\subsubsection {Data Parallel Problems}

for problems where each task generates an independent piece of data (ex/pixels in an image), we wish to enforce the following constraint:\\\\
\textbf{Strong Assertion}: All tasks meet a result constraint.\\ $\forall t_i \in T, t_i \rightarrow m_j ~|~ (\int_{-\epsilon}^{\epsilon} P_j(s_i+ \epsilon)) > P_{spec}$\\
ex: black-scholes, we want to ensure each price is likely to give a correct answer\\\\
\textbf{Weak Assertion}: On average, all tasks meet a result constraint.\\ $t_i \rightarrow m_j ~|~ \frac 1 {n} \sum_{i=1}^n (\int_{-\epsilon}^{\epsilon} P_j(s_i+ \epsilon)) > P_{spec}$\\
ex: image generation, allow individual pixels to be arbitrarily bad as long as entire image is on average, ok.\\\\
\textbf{Issues}: we don't know the error distribution relative to the solution beforehand, we would need to build a per-machine error distribution during runtime. We can empirically estimate $1-\int_{-\epsilon}^{\epsilon} P_j(s_i+ \epsilon))$ at runtime by re-executing tasks at runtime and taking the number of incorrect results over the total number of results. We will call this $p_j^f$, the probability of task failure for machine j.
\subsection {Assumptions}
\textit{Martin: No batching then batching. Bad machine generates fault result U(-infty, infty), actual result is N(0,1)}
\begin{itemize}
	\item \textit{No Batching}: one task execution on one machine, followed by one reduction. 
	\item \textit{No Catastrophic Failures}: Typically if you send too many tasks to a machine, it may crash. We won't model this for now.
	\item \textit{Per-Item Outlier Detection}: Simple per-primitive outlier detector - no cross-item correlation.
	\item \textit{User provides reliability specification}: user provides $P_{spec}$ and $\epsilon$ for each data element in each task set.
\end{itemize}

\subsection {Parameter Space}
Topaz has the following parameters to play with.\\\\
\textbf{Scheduling:} [cost:$c_j$] We can choose $j : t_i \rightarrow m_j,m_r$:\\
\textbf{Outlier Detection:} We can configure different parameters of the online outlier detector. For our basic outlier detector, that is the number of standard deviation we can use.\\
\section {Naive Model}
More Assumptions:\\
\begin{itemize}
	 \item \textit{We're discussing a taskset with a single numerial output.}
	\item \textit{Output Distribution follows Normal Distribution}: A simplification. 
	\item \textit{Data Parallel, Strong Specification}: provide a hard desired probability $p_{spec}$ falls within $\epsilon$
\end{itemize}
\subsubsection {Scheduling}
We greedily choose the smallest $c_j$ where $1-p_j^f$ (the probability of a particular element being outside of $s_i \pm \epsilon$) meets $p_{spec}$
\subsubsection {Outlier Detector} 
Online detector that models the output distribution as $N(\mu, \sigma)$. This does not work well for output distributions that do not resemble normal distributions. If an outlier detector detects an outlier, for now, always re-execute.\\
\textbf{Re-execution:} [cost:$c_j + c_r$] We can get a completely correct result and improve the outlier detector on outlier detection.\\
\textit{Parameters:}\\
\begin{itemize}
	\item K $\rightarrow$ the number of standard deviations from mean to consider ok.
\end{itemize} 

\textit{Martin analysis: lets take a particular computation where we add up all the numbers first. Outlier detector algorithm (normal distribution), Prove some bound on how results are likely to be.} 
\textit{Martin: we will eventually need to include task work - lets leave task work out.}
Issues we will eventually have to deal with.
\begin{itemize}
	\item Amount of work in each task (ie tasks have variable work quantities)
	\item Batching 
	\item Crashes - outlier detection on task times
	\item 2 kinds of re-executions: pitch result and re-execute. execute someplace else.
\end{itemize}
Outlier detection strategies:
1. main processor can send outlier data to child machine to remote machine.
\section {Example: Array Summation}
Assume
\begin{itemize}
	\item we $\sum$ the results of T tasks, where $t: i \rightarrow r$ and $r \in N(\mu, \sigma)$.
	\item we have an outlier detector $N(\mu, k\sigma)$ that predicts the output distribution and considers results within k standard deviations ok, where k is unknown. 
	\item on outlier detect, we re-execute
	\item we have a desired percent error of $p$
	\item assume $w$ percent of tasks fail, Assume the output distribution is uniformly distributed $U(F_{min}, F_{max})$.
\end{itemize}
$p = \frac {S_{urel} - S_{rel}} {S_{rel}} = \frac {\sum r_{urel} - r_{rel}} {\sum r_{rel}}$\\
All undetected outliers fall within $(\mu - k \sigma, \mu + k \sigma)$. In the worst case: $N(\mu, k\sigma) \rightarrow |r_{urel} - r_{rel}| \sim 2 k\sigma$\\
With increasing k, the percent of undetected outliers increases:\\
$\int_{\mu-k\sigma}^{\mu+k\sigma} \frac 1 {F_{max} - F_{min}}$\\
$[(\mu+k\sigma - \mu + k\sigma) \frac 1 {F_{max} - F_{min}}]$\\
$[\frac {2 k \sigma} {F_{max} - F_{min}}]$\\
\\
We can approximate the absolute error as follows:\\
$\sum r_{urel} - r_{rel} \approx w \cdot |T| \cdot \frac {2k \sigma} {F_{max} - F_{min}} \cdot k \sigma = w \cdot |T| \cdot \frac {2k^2 \sigma^2} {F_{max} - F_{min}}$\\
We can approximate the average sum as follows:\\
$\sum r_{rel} = |T| \mu$\\\\
We rewrite p as:\\
$p \approx \frac {w \cdot |T| \cdot 4k^2 \sigma^2} {(F_{max} - F_{min}) \cdot |T| \mu}$\\
$p \approx \frac {w \cdot 4k^2 \sigma^2} {(F_{max} - F_{min}) \mu}$\\
$\frac {(F_{max} - F_{min})\mu \cdot p} {w \cdot 2 \sigma^2}  = k^2$\\\\
Therefore, the ideal number of standard deviations to consider, given a desired output error rate, is:\\
$k = \sqrt{\frac {(F_{max} - F_{min})\mu \cdot p} {w \cdot 4 \sigma^2}}$\\\\
Obviously (1) we don't know the error distribution, nor does (2) our outlier detector perfectly fit the output distribution. The the rate of task errors for a particular machine is not known.


\end{document}  